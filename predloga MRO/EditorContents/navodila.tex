\extrachap{Navodila za izdelavo druge seminarske naloge}
\begin{itemize}
\item Vaše datoteke se nahajajo v direktorijih \texttt{Skupina*}, kjer \texttt{*} predstavlja številko vaše skupine - glavna datoteka je \texttt{main.tex}.
\item Slike shranjujte v svoj direktorij.
\item Vse labele začnite z znaki \texttt{g*:}, kjer \texttt{*} predstavlja številko vaše skupine.
\item Pri navajanju virov uporabite datoteko \texttt{references.bib}, ki se nahaja v korenskem direktoriju projekta.
\end{itemize}

V seminarski nalogi se boste ukvarjali z načrtovanjem in analizo omrežij in z INET ogrodjem simulacijskega okolja OMNeT++. INET ogrodje je dostopno na naslovu \url{http://inet.omnetpp.org/}.

Oddana seminarska naloga naj vsebuje:

\begin{itemize}
	\item izvorne datoteke poročila v LaTeXu (poročilo naj vsebuje približno 15 strani),
	\item poročilo v pdf formatu,
	\item projekt z zgledi, ki naj vsebujejo razlago in komentarje,
	\item prosojnice za predstavitev (predstavitev naj traja približno 10 minut).
\end{itemize}

Vaše poglavje seminarja (uporabite predlogo v orodju LaTeX) naj vsebuje:

\begin{enumerate}
	\item uvod,
	\item zahtevane točke seminarja (priporočamo vnos enega razdelka oz. section za vsako točko),
	\item opis izbranih parametrov in analiza njihovih vrednosti v vašem modelu,
	\item analiza rezultatov,
	\item zaključek in končni komentarji
\end{enumerate}

Vsak seminar predstavlja eno poglavje v knjigi \textit{Modeliranje in simulacija računalniških omrežij v orodju OMNeT++}. 

Delo bo potekalo v skupinah s štirimi študenti. Za skupinsko delo uporabljajte repozitorij, kot je npr. dropbox\footnote{\url{https://www.dropbox.com}} ali git\footnote{\url{https://bitbucket.org}}. Cilj seminarja so izdelki in poročila, ki jih lahko ob zaključku združimo v celoto (glej rezultate seminarjev iz prejšnjih let). Poročilo pišite v okolju LaTeX, kjer lahko za lažje skupinsko delo uporabljate okolje, kot je npr. overleaf\footnote{\url{https://www.overleaf.com/}}. Za iskanje virov uporabljajte iskalnike znanstvene literature\footnote{\url{https://scholar.google.si}, \url{www.sciencedirect.com}, \url{https://www.scopus.com}, \url{https://arxiv.org}, \url{http://citeseerx.ist.psu.edu}}. Rok za oddajo seminarja je 16. 1. 2025. Predstavitve nalog bodo 17. 1. 2025. Na predstavitvi bo imela vsaka skupina 10 minutno predstavitev svojega izdelka, nato bo sledila krajša diskusija.

Z ustreznimi viri (literaturo) polnite vašo BIB datoteko. Poročilo redno izpopolnjujte. Ker gre za skupinsko delo, mora biti na koncu poročila tudi poglavje z naslovom Doprinosi avtorjev, kjer v enem stavku zapišite, kakšen je bil doprinos posameznega člana skupine.


